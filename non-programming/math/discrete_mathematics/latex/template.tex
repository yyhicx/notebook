\documentclass{ctexart}

\usepackage{amsmath}
\usepackage{amssymb}
\usepackage{subfigure}
\usepackage[graphicx]{realboxes}
\counterwithin{figure}{section}

\begin{document}

\section{归谬法}
\flushleft{设要证明$H_1 \land H_2 \land ... \land H_n \Rightarrow C$,其中$H_1$、$H_2$、...、$H_n$和C是命题公式。}
\begin{align*}
&\text{令:} &S &\Leftrightarrow H_1 \land H_2 \land ... \land H_n \\
&\text{则上式可以简化为:} &S &\Rightarrow C \\
&\text{由蕴涵的定义有:} &1 &\Leftrightarrow S \rightarrow C \Leftrightarrow \lnot S \lor C \\
&\text{两边否定:} &0 &\Leftrightarrow S \land \lnot C \Leftrightarrow H_1 \land H_2 \land ... \land H_n \land \lnot C
\end{align*}
\flushleft{上式等价下面两式:}
\begin{align*}
&0 \Rightarrow H_1 \land H_2 \land ... \land H_n \land \lnot C &\text{(显然成立)} \\
&H_1 \land H_2 \land ... \land H_n \land \lnot C \Rightarrow 0
\end{align*}
\flushleft{所以要证明:$H_1 \land H_2 \land ... \land H_n \Rightarrow C$,只需要证明:}
\begin{align*}
H_1 \land H_2 \land ... \land H_n \land \lnot C \Rightarrow 0
\end{align*}
\flushleft{其中,$\lnot C$叫做附加前提。这种间接推理方法称为归谬法,也称为反证法。}

\newpage

\section{CP规则}
\flushleft{有时要证明的有效结论是一个条件命题,即要证明:}
\begin{align*}
H_1 \land H_2 \land ... \land H_n \Rightarrow (A \rightarrow B)
\end{align*}
\begin{align*}
&\text{令:} &S & \Leftrightarrow H_1 \land H_2 \land ... \land H_n \\
&\text{则上式可以简化为:} &S & \Rightarrow (A \rightarrow B) \\
&\text{由蕴涵的定义有:} &1 &\Leftrightarrow S \rightarrow (A \rightarrow B) \Leftrightarrow \lnot S \lor (\lnot A \lor B) \\
& & &\Leftrightarrow (\lnot S \lor \lnot A) \lor B \Leftrightarrow \lnot (S \land A) \lor B \\
& & &\Leftrightarrow (S \land A) \rightarrow B \\
& & &\Leftrightarrow H_1 \land H_2 \land ... \land H_n \land A \rightarrow B
\end{align*}
\flushleft{即 $\quad H_1 \land H_2 \land ... \land H_n \land A \Rightarrow B$}
\flushleft{所以,要证明$H_1 \land H_2 \land ... \land H_n \Rightarrow (A \rightarrow B)$,只需要证明:}
\begin{align*}
H_1 \land H_2 \land ... \land H_n \land A \Rightarrow B
\end{align*}
\flushleft{其中A叫做附加前提。这种间接推理方法称为CP规则。}

\newpage

\section{命题表示的例题}
\flushleft{第一题,用谓词表示命题:}
\begin{align*}
\text{对任意整数x,}x^2 - 1 = (x+1)(x-1)\text{是恒等式}
\end{align*}
\flushleft{令$I(x): x$是整数,$f(x)=x^2-1$,$g(x)=(x+1)(x-1)$,$E(x, y): x = y$,则该命题可表示为:}
\begin{align*}
(\forall x)(I(x)\rightarrow E(f(x), g(x)))
\end{align*}
\flushleft{第二题,将命题“没有最大的自然数”符号化。}
\flushleft{解析:命题中“没有最大的”显然是对所有的自然数而言,所以可理解为“对所有x,如果x是自然数,则一定还有比x大的自然数”,再具体点,即“对所有的x如果x是自然数,则一定存在y,y也是自然数,且y比x大”。}
\flushleft{令$N(x): x$是自然数,$G(x, y): $x大于y,则原命题表示为:}
\begin{align*}
(\forall x)(N(x)\rightarrow (\exists y)(N(y) \land G(y, x)))
\end{align*}

\newpage

\section{谓词演算的等价式和蕴涵式}
\flushleft{1.量词否定等价式:}
\begin{align*}
\lnot(\forall x)A \Leftrightarrow (\exists x)\lnot A \\
\lnot(\exists x)A \Leftrightarrow (\forall x)\lnot A
\end{align*}
\flushleft{2.量词辖域缩小或扩大等价式:}
\flushleft{设B是不含x自由出现,A(x)为有x自由出现的任意公式,则有:}
\begin{align*}
(\forall x)(A(x) \land B) \Leftrightarrow (\forall x)A(x) \land B \\
(\forall x)(A(x) \lor B) \Leftrightarrow (\forall x)A(x) \lor B \\
(\forall x)(A(x) \rightarrow B) \Leftrightarrow (\exists x)A(x) \rightarrow B \\
(\forall x)(B \rightarrow A(x)) \Leftrightarrow B \rightarrow (\forall x)A(x) \\
(\exists x)(A(x) \land B) \Leftrightarrow (\exists x)A(x) \land B \\
(\exists x)(A(x) \lor B) \Leftrightarrow (\exists x)A(x) \lor B \\
(\exists x)(A(x) \rightarrow B) \Leftrightarrow (\forall x)A(x) \rightarrow B \\
(\exists x)(B \rightarrow A(x)) \Leftrightarrow B \rightarrow (\exists x)A(x)
\end{align*}
\flushleft{3.量词分配律等价式:}
\flushleft{设A(x),B(x)为有x自由出现的任何公式:}
\begin{align*}
(\forall x)(A(x) \land B(x)) \Leftrightarrow (\forall x)A(x) \land (\forall x)B(x) \\
(\exists x)(A(x) \lor B(x)) \Leftrightarrow (\exists x)A(x) \lor (\exists x)B(x)
\end{align*}
\flushleft{4.蕴涵式:}
\begin{align*}
&(\forall x)P(x) \Rightarrow (\forall x)P(x) \lor (\exists y)Q(y) &\text{附加规则} \\
&((\forall x)P(x) \rightarrow Q(x, y)) \land (\forall x)P(x) \Rightarrow Q(x, y) &\text{假言推理} \\
&(\forall x)A(x) \lor (\forall x)B(x) \Rightarrow(\forall x)(A(x)\lor B(x)) \\
&(\exists x)(A(x) \land B(x)) \Rightarrow (\exists x)A(x) \land (\exists x)B(x)
\end{align*}

\end{document}
